\documentclass[a4paper,11pt,oneside]{article}
\usepackage[dvipdfm]{graphicx}
\usepackage{bmpsize}
\usepackage{pgfplots}
\usepackage{amsmath}
\usepackage{amssymb}
\usepackage{amsthm}
\usepackage{graphicx}
\author{Bryan Kaperick}

\newcommand{\norm}[1]{\|#1\|}

\begin{document}



\theoremstyle{definition}
\newtheorem{definition}{L+2 Position Definition}
\newtheorem{lemma}{3-Row Chomp Lemma}

\begin{definition}
An ``$L+2$ Position'' is a chomp board which has $x$ cookies in its first row, and two cookies in its second row, and one cookie in its third row.  Note that since chomp boards have rows of non-increasing length, the condition that the third row has one cookies implies that all rows from one to its final row, $y$, have only one cookie.  These positions can be characterized by their width, $x$, and their height, $y$.
\end{definition}

\begin{lemma}
	\label{lem:3r}
	For a chomp board of three rows with two cookies in its third row, it is a P-position iff the difference between the first and second rows is two cookies.
\end{lemma}

\newtheorem{transposelemma}{Transpose Lemma}

\begin{transposelemma}
	\label{lem:tr}
	For any arbitrary Chomp board $B$, $B$ is a P position iff $B^T$ is a P position.
\end{transposelemma}


\begin{proof}
	Suppose some board $B$ is a P position.  By definition, every reachable board is then an N position.  Every reachable board is a result of some move $(i,j)$ where $1\leq i \leq m$ and $1\leq j \leq n$ where $m$ is the number of rows of the board and $n$ is the number of cookies in the $i$th row of the board.  This move takes the cookies denoted $C_{ij}$ along with every cookie such that removing any cookie $C_{i_oj_o}$ removes all cookies $C_ij$ such that $i_o \leq i\leq m$ and $j_o\leq j \leq n$.  $B^T$ is defined as the matrix of cookies such that each cookie $C_ij$ in $B$ is mapped to $C_ji$ in $B^T$.  So, removing any cookie $C_{j_oi_o}$ in $B^T$ removes all cookies   $C_ji$ such that $j_o\leq j \leq n$ and $i_o \leq i\leq m$.  Since every move in $B$ corresponds to a moves in $B^T$, a player could counteract any possible move in $B$ by removing the corresponding cookie in $B^T$.  This implies every reachable position in $B^T$ is an N position, so $B^T$ is a P position.  Without loss of generality, it can be shown that $B^T$ being a P position implies that $B$ is a P position.  Since these two statements imply each other, we can conclude that $B$ is a P position iff $B^T$ is a P position.
\end{proof}



\newtheorem{theorem}{ ``L+2 Position'' Theorem}
\begin{theorem}
	Let $B$ be an $L+2$ Position chomp board of width $w$ and height $h$.
	\begin{enumerate}
		\item $(w = h + 1) \,\land\, (w$ is even $)\,\land\,( h \geq 3) \implies (B$ is a P-Position$)$
		\item $(w = h - 1) \,\land\, (h$ is even $)\,\land\,( h \geq 6) \implies (B$ is a P-Position$)$
		\item All other $L+2$ Positions are N-Positions.
	\end{enumerate}
\end{theorem}<++>

\begin{proof}
This is a proof by induction on $n$.\\
Let $B_a$ be an L+2 Position chomp board of width $2n$ and height $2n-1$ for some $n \geq 2$.  Let $B_b$ be an L+2 Position chomp board of width $2m-1$ and height $2m$ for some $m \geq 3$.  Let $n = 2$.  Then, $B_a$ is an $L+2$ Position board of width $4$ and height $3$.  Using \ref{lem:3r}, this is a P board, so the base case for $B_a$ holds.  Now let $m = 3$.  Then $B_b$ is an $L+2$ Position board of width $5$ and height $6$.  This can (through the process of elimination) seen to be a P board.  Let $k$ be an integer such that $2 \leq k \leq n$ and $l$ be an integer such that $3 \leq l \leq m$.  We will assume that all L+2 Position boards of width $2k$ and height $2k-1$ are P boards (Inductive assumption A) and that all L+2 Position boards of width $2l-1$ and height $2l$ are also P positions (Inductive Assumption B).\\
Consider a board of width $2(n+1)$, or $2n+2$ and height of $2(n+1)-1$, or $2n + 1$.  There are a several cases for which move that player one (P1) can make.  Applying proof by contradiction, assume these boards are N boards.  Then, there exists at least one reachable board which will be an P board.  Examining the possible cases, the following results are gathered using the  notation that ``$P1(x,y)$'' means that player one removes the $x$th cookie in row $y$.
\begin{itemize}
	\item [$P1(1, 3)$] This move converts the board to a 2-row chomp board, which has been solved for all cases.  Since the first row has an even number of cookies, it will never have exactly one more cookies than the second row, so this must be an N board.
	\item [$P1(1,5)$] P2 can respond by playing $P2(3,1)$. By \ref{lem:tr}, each board is identical to its transpose, so this 2-column board is identical to the transposed 2-row board.  Then, since the first row is one greater than the second, it must be a P board.  Then, that means P2 was handed an N board, so after $P1(1,5)$ this board must be an N board. 
	\item [$P1(1, 2k)$] Then, player 2 (P2) can respond to P1 by playing $P2(2k+1,1)$, which, by Inductive Assumption A will convert this board to a P board.  Then, that means P2 was handed an N board, so after $P1(1,2k)$ this board must be an N board.
	\item [$P1(1, 2k-1)$] Then, player 2 (P2) can respond to P1 by playing $P2(2k,1)$, which, by Inductive Assumption B will convert this board to a P board.  Then, that means P2 was handed an N board, so after $P1(1,2k-1)$ this board must be an N board.
	\item [$P1(2,2)$] This move converts the board into an unbalanced L position, which is known to be an N board.
	\item [$P1(2, 3)$] $L+1$ positions have been completely classified, and this position is known to be an N board.
	\item [$P1(1, 2n+1)$] If P2 responds with $P2(2n+1, 1)$, which is a board of equal height and width that is not a balanced L position.  This is known to be a P board, so that means P2 was handed an N board, so after $P1(1, 2n+1)$ the board must be an N board.
	\item [$P1(3, 1)$] By \ref{lem:tr}, each board is identical to its transpose, so this 2-column board is identical to the transposed 2-row board.  Then, since the first row is not exactly one greater than the second, it must be an N board.
	\item[$P1(4,1)$] Each board is identical to its transpose, so this 3-column board is identical to the transposed 3-row board.  There are exactly two 3-row boards that are P positions with 1 cookies in their third row, and this is not one of them, so it is an N board.
	\item[$P1(5, 1)$] Then this can be followed by $P2(1,4)$ which is assumed to be a P board as it is the base case for inductive assumption A.  Since P2 is able to move and make the board a P board, P2 was handed an N board.
	\item [$P1(6,1)$] P2 can move $P2(1,7)$ which, using Inductive Assumption B, is a P board.  Since P2 is able to move and make the board a P board, P2 was handed an N board.
	\item [$P1(2k-1, 1)$] Note here that $k\geq 3$ since all cases below that have been explicitly examined above.  Then, P2 can respond with $P2(1, 2k-2)$ which, by Inductive Assumption A, is a P board.  Since P2 is able to move and make the board a P board, P2 was handed an N board.
	\item [$P1(2k, 1)$] Again, assume $k\geq 3$ since all lower cases have been explicitely examined above.  Then, P2 can respond with $P2(1,2k+1)$, which is a P board using Inductive Assumption B.  Since P2 is able to move and make the board a P board, P2 was handed an N board.
	\item [$P1(2n+2, 1)$] Since the width and height of the board are the same, P2 can respond $P2(2, 2)$ and give P1 a balanced L position, which is a P board.  Since P2 is able to move and make the board a P board, P2 was handed an N board.
\end{itemize}
	Since in all possible cases, P1 must hand P2 an N board, we can conclude that our assumption that an L+2 board with width of $2n+2$ and height of $2n+1$ is an N position was incorrect.  Then, it must be a P position, assuming that Inductive Assumption B holds.  To prove this, we will next examine a board of width $2(n+1)-1$, or $2n+1$ and height $2(n+1)$, or $2n+2$.\\
	To be continued\dots

\end{proof}

\newtheorem{2r1theorem}{ 2r-1/2c-1 Theorem}

\begin{2r1theorem}
	Let $B$ be an arbitrary chomp board of $r$ rows, $c$ columns and $n$ cookies.
	\begin{enumerate}
		\item 	$n\geq 2c-1$ and $n\geq 2r-1$
		\item	$(n=2c-1)\land(n=2r-1) \implies B$ is a P position
	\end{enumerate}
\end{2r1theorem}


\end{document}
